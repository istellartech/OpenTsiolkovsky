\documentclass[12pt]{jsarticle}
\usepackage{here}
\usepackage[dvipdfmx]{graphicx}
\usepackage{wrapfig}
\usepackage{listings}

\title{OpenTsiolkovsky テクニカルドキュメント}
\author{Interstellar Technologies Inc. 稲川貴大}
\date{\today}


\begin{document}
% \begin{titlepage}
\maketitle
\begin{abstract}

\end{abstract}
% \thispagestyle{empty}
% \end{titlepage}
% \tableofcontents
% ここから文章を書いて下さい。

\tableofcontents

\newpage
\section{イントロダクション}
OpenTsiolkovskyはロケット用フライトシミュレータです。
地球の重力のみ考慮した、地球近傍でのロケットの軌道投入までの短時間の
軌道計算を行なうことができます。
地球中心の慣性座標系で計算を行います。

ロケットの概念設計から、ミッション決定、既存のロケットの能力判別など、
ロケットの軌道計算は様々なファイズで必要とされます。

軌道計算は6自由度の運動方程式で記述可能です。並進3自由度(XYZ)、
回転3自由度(ロールピッチヨー)です。
基礎的な情報しか必要無いときは自由度を減らし、1自由度、2自由度、3自由度、
5自由度など複数のシミュレーションで十分ですが、OpenTsiolkovskyでは
6自由度で計算しています。

運動方程式は常微分方程式となります。
この常微分方程式のパラメータになるものをユーザが入力し、
これをソルバー(OpenTsiolkovskyの機能)によって数値積分したものを
出力します。

つまり、OpenTsiokovskyとはロケットのパラメータファイルを
パースして取り込み、常微分方程式を数値積分で解き、結果を整形して
出力するソフトウエアです。

ちなみにTsilkovskyとはロシア・ソ連のロケット研究者で、
ツォルコフスキーの式と呼ばれるロケットの基礎的な理論を示した人物。
名言としては下記があります。

「今日の不可能は、明日可能になる」

「地球は叡智の揺り籠だが、しかしその叡智が永遠に揺り籠に留まるべきではない。」

\begin{equation}
\Delta V = I_{sp} g_{0} \ln \frac{m_0}{m_T}
\end{equation}


\section{Program capabilities/Installation}

\section{Basis of theory and model}

\section{Input definition}

\section{Output definition}

% ここまで文章を書いて下さい。
\end{document}
